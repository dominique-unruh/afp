\documentclass[11pt,a4paper]{report}
\usepackage[utf8]{inputenc}
\usepackage[T1]{fontenc}
\usepackage{isabelle,isabellesym}
\usepackage{amssymb}
% this should be the last package used
\usepackage{pdfsetup}

\renewcommand{\isadigit}[1]{\ensuremath{#1}}

% font size
\renewcommand{\isastyle}{\isastyleminor}

%See LaTeXsugar2
\newlength{\funheadersep}
\setlength{\funheadersep}{\smallskipamount}

% urls in roman style, theory text in math-similar italics
\urlstyle{rm}
\isabellestyle{literal}

% no right margin in quote:
\renewenvironment{quote}
{\list{}{}%
\item\relax}
{\endlist}

\newcommand{\eqnum}[1]{{\upshape\refstepcounter{equation}\hfill(\theequation\label{#1})}}

\begin{document}

\title{Alpha-Beta Pruning}
\author{Tobias Nipkow\\Technical University of Munich}
\maketitle

\begin{abstract}
Alpha-beta pruning is an efficient search strategy for two-player game trees.
It was invented in the late 1950s and is at the heart of most implementations
of combinatorial game playing programs. These theories formalize and verify a number of
variations of alpha-beta pruning, in particular fail-hard and fail-soft,
and valuations into linear orders, distributive lattices and domains with negative values.

A detailed presentation of these theories can be found in the chapter \emph{Alpha-Beta Pruning}
in the (forthcoming) book
\href{https://functional-algorithms-verified.org/functional_data_structures_algorithms.pdf}{Functional Data Structures and Algorithms --- A Proof Assistant Approach}.
\end{abstract}

{\renewcommand{\isanewline}{\\}
\input{Alpha_Beta_Overview}
}
\bibliographystyle{abbrv}
\bibliography{root}
\newpage
\tableofcontents
\newpage
\input{Alpha_Beta_Linear}
\input{Alpha_Beta_Lattice}

\end{document}

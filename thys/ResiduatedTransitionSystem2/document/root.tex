\documentclass[11pt,notitlepage,a4paper]{report}
\usepackage[T1]{fontenc}
\usepackage{isabelle,isabellesym,eufrak}
\usepackage{amssymb,amsmath}
\usepackage[english]{babel}

% this should be the last package used
\usepackage{pdfsetup}

% urls in roman style, theory text in math-similar italics
\urlstyle{rm}
\isabellestyle{it}

\begin{document}

\title{Residuated Transition Systems II:\\ Categorical Properties}
\author{Eugene W. Stark\\[\medskipamount]
        Department of Computer Science\\
        Stony Brook University\\
        Stony Brook, New York 11794 USA}
\maketitle

\begin{abstract}
This article extends the formal development of the theory of residuated transition systems
(RTS's), begun in the author's previous article, to include category-theoretic properties.
There are two main themes: (1) RTS's {\em as} categories; and (2) RTS's {\em in} categories.
Concerning the first theme, we show that every RTS determines a category via the
``composite completion'' construction given in the previous article,
and we obtain a characterization of the ``categories of transitions'' that arise in this way.
Concerning the second theme, we show that the ``small'' extensional RTS's having arrows
that inhabit a type with suitable closure properties, form the objects of a cartesian closed
category {\bf RTS} when equipped with simulations as morphisms.
This category can in turn be regarded as contained in a larger, $2$-category-like structure
which is a category under ``horizontal'' composition, and is an RTS under ``vertical''
residuation.  We call such structures {\em RTS-categories}.
We explore in particular detail the RTS-category ${\bf RTS}^\dagger$ having RTS's as objects,
simulations as $1$-cells, and transformations between simulations as $2$-cells.
As a category, ${\bf RTS}^\dagger$ is also cartesian closed, and the category {\bf RTS}
occurs as the subcategory comprised of the arrows that are identities with respect to the
residuation.
To obtain these results various technical issues, related to the formalization within the
relatively weak HOL, have to be addressed.
We also consider RTS-categories from the point of view of enriched category theory
and we show that RTS-categories are essentially the same thing as categories enriched in {\bf RTS},
and that the RTS-category ${\bf RTS}^\dagger$ is determined up to isomorphism by its
cartesian closed subcategory {\bf RTS}.
\end{abstract}

\newpage

\phantomsection
\addcontentsline{toc}{chapter}{Contents}
\tableofcontents

\phantomsection
\addcontentsline{toc}{chapter}{Introduction}
\chapter*{Introduction}

This article continues the formal development of the theory of residuated transition systems
(RTS's) which was begun in the previous article \cite{ResiduatedTransitionSystem-AFP}.
A particular theme of the present article is the development of category-theoretic properties.
These were intentionally omitted from the previous article in order to avoid dependence of
the basic RTS theories on a development of category theory.
The present article has two main themes: (1) considering RTS's as themselves being
(or perhaps generating) categories; and (2) studying RTS's as objects within categories.
With respect to the first theme, we recall from the previous article that every RTS determines
an extensional RTS with composites -- its ``composite completion.''  As a structure consisting
of a set of arrows equipped with a partial composition that is associative and satisfies
left and right identity laws, an RTS with composites can obviously be regarded as a category.
In view of the fact that the composition is derived from an underlying residuation operation,
it is straightforward to show that such a category has unique ``bounded pushouts'';
that is, a unique pushout exists for every span that is ``bounded'' in the sense that it
can be completed to a commutative square.  In addition, in such a category, every arrow
is an epimorphism and there are no nontrivial isomorphisms.
We define a ``category of transitions'' to be a category with these properties, and we show
that every such category is in fact an extensional RTS with composites, where consistency
of a span of transitions coincides with boundedness and the residuation is obtained from
pushouts.

Concerning the second theme, we are interested a category {\bf RTS} whose objects are
extensional RTS's and whose morphisms are simulations between such RTS's.
Since the set of morphisms between two extensional RTS's $A$ and $B$ is itself an extensional
RTS having as its morphisms the transformations between simulations, it is clear that
{\bf RTS} has additional structure to be explored here beyond that of a simple category.
As we expect the category {\bf RTS} (or closely related structures) to be the main focus
of attention in applications of residuated transition systems (to programming language
semantics, for example), our objective is to clarify this structure as much as possible and
to set up technology for working formally with this category.
There are some technical problems that arise with the formalization of this material in
the context of Isabelle/HOL, though.  First of all, RTS's exist with arrows at arbitrary types,
so it is impossible in HOL to directly formalize a category whose objects encompass
``all'' RTS's.  Instead, we have to consider categories of RTS's whose arrows inhabit
some particular type, though we may exploit polymorphism to prove general theorems that hold
for any such category.  Secondly, the fact that the hom-sets of a category of extensional RTS's
and simulations themselves admit the structure of an extensional RTS suggests that we ought
to be looking at the category {\bf RTS} as a {\em closed} category;
in fact cartesian closed, as it happens.  Here again, we face some limitations posed by our
choice to carry out the formalization within Isabelle/HOL.  In particular, a cartesian
closed category, having RTS's as objects and as homs the sets of all simulations between such,
cannot be constructed in pure HOL, unless we restrict our attention to finite RTS's only.
However, we can work around this limitation if we are willing to extend pure HOL with the
assumption that there is a ``universe'' type that is ``large'' enough to be closed under the
function space constructions that we need to perform in order to achieve cartesian closure.
The Isabelle HOL library already has a well-developed theory of this kind; namely, the
{\em ZFC\_in\_HOL} theory, which axiomatically extends HOL with a type $V$ and a notion of
``smallness'', such that the small sets of type $V$ satisfy the axioms of ZFC.
In our development, we presuppose the existence of a notion of smallness and suppose that
the underlying arrow type of the category {\bf RTS} is closed under the construction of
small function spaces, but we have avoided ``hard coding'' the particular type $V$ defined
in {\em ZFC\_in\_HOL} into our development.  This independence from a particular
choice of ``universe'' comes at a cost, however: in order to show that the category
{\bf RTS} admits type-increasing constructions such as products and exponentials,
we have to concern ourselves in each case with finding a suitable ``type-reducing map''
to show that the RTS's constructed with arrows at the higher types are in fact isomorphic
to RTS's that already exist at the original arrow type.  We note that these complications
only arise in showing that {\bf RTS} admits various categorical constructions --- once this
has been done we can work with these constructions in a simple way using the universal
properties that characterize them.

To summarize the above, one of our main results is the construction of a cartesian closed
category {\bf RTS}, whose objects are in bijection with ``small'', extensional RTS's whose
arrows inhabit a suitable ``universe'' type $\alpha$ and each of whose hom-sets ${\bf RTS}(A, B)$
is in bijective correspondence with the set of all simulations between $A$ and $B$.
We show that the particular type $V$ axiomatized in {\em ZFC\_in\_HOL} satisfies the requirements
for the universe type $\alpha$, but our development does not otherwise depend on details of
{\em ZFC\_in\_HOL} except for the notion of smallness defined therein.
We prove theorems that allow us to pass back and forth between notions internal to {\bf RTS}
and corresponding external notions expressed in terms of the concrete structure of
RTS's and simulations.

The fact that {\bf RTS} is cartesian closed is a consequence of the fact that the
transformations between simulations from an extensional RTS $A$ to an extensional RTS $B$
may themselves be regarded as the arrows of an exponential RTS $[A, B]$.
The category {\bf RTS} may therefore be regarded as a category ``enriched in itself''
\cite{kelly-enriched-category}.  However, it seems more natural to think of {\bf RTS}
as something more like a $2$-category, where the $0$-cells correspond to RTS's,
the $1$-cells correspond to simulations, and the $2$-cells correspond to transformations
between simulations.  Unless we restrict ourselves {\em a priori} to RTS's with composites
(something that we do not wish to do), the resulting structure will not actually be
a $2$-category, because the homs will in general be RTS's that do not necessarily
admit composition of transitions ({\em i.e.}~of transformations).
So instead the kind of structure we obtain consists of a category under ``horizontal''
composition, and an RTS under ``vertical'' residuation.
We formalize such a structure, calling it an ``RTS-category''.
We show that there is an RTS-category ${\bf RTS}^\dag$, whose $0$-cells (objects) are in
bijection with the small, extensional RTS's with arrows at a universe type $\alpha$,
whose $1$-cells (arrows) are in bijection with simulations between such RTS's,
and whose $2$-cells are in bijection with transformations between such simulations.
As a category, ${\bf RTS}^\dag$ is itself cartesian closed, and the subcategory defined
by the $1$-cells (which coincide with the identities of the residuation) is the
ordinary cartesian closed category ${\bf RTS}$.
We prove results that allow us to pass back and forth between notions
internal to ${\bf RTS}^\dag$ and the corresponding external notions.
The construction of ${\bf RTS}^\dag$ and the proof of associated facts constitutes a second
group of main results of this article.

Finally, our third group of main results concerns the clarification of the relationship
between the notion of RTS-category and that of a category enriched in {\bf RTS}.
We show that from a category $E$ enriched in {\bf RTS} we
can construct an RTS-category $C$ having as its set of $2$-cells the disjoint union of
the sets of arrows of the RTS's underlying the hom-objects of $E$.  Conversely, given an
RTS-category $C$ we can construct a corresponding category $E$ enriched in {\bf RTS}
by taking as the ``hom-objects'' of $E$ the objects of {\bf RTS} corresponding to the
``hom-RTS's'' of $C$.
These correspondences are functorial and extend to an equivalence between a category
of RTS-categories and a category of {\bf RTS}-enriched categories (for a suitable
definition of morphism in each case).  So, RTS-categories and categories enriched in
{\bf RTS} amount to the same thing, though the definition of RTS-categories is more elementary
and will likely be easier to work with in applications.

The remainder of this article is organized as follows:
In Chapter \ref{preliminaries_chapter}, we have proved various facts we need about RTS's,
simulations, and transformations, which are not part of the previous article
\cite{ResiduatedTransitionSystem-AFP}.
In Chapter \ref{rts_as_categories_chapter}, we present the results discussed above
which pertain to the theme ``RTS's as Categories''.
Chapter \ref{rts_constructions_chapter} defines various concrete constructions on RTS's,
including product and exponential, and proves associated universal properties.
In addition, this section defines the constraints on a type $\alpha$ required for it
to serve as a ``universe'' and establishes related facts.
Finally, in Chapter \ref{rts_in_categories_chapter}, we define the notion of
RTS-category, construct the RTS-category ${\bf RTS}^\dag$ and prove facts about it,
including cartesian closure, construct the subcategory {\bf RTS} and prove facts about
it as well, and finally establish the equivalence of RTS-categories and categories
enriched in {\bf RTS}.

\chapter{Preliminaries}
\label{preliminaries_chapter}

This section develops some extensions to theories contained in the previous AFP
articles \cite{ResiduatedTransitionSystem-AFP} and \cite{MonoidalCategory-AFP}.

  \input{Preliminaries.tex}

\chapter{RTS's as Categories}
\label{rts_as_categories_chapter}

As shown in the previous article \cite{ResiduatedTransitionSystem-AFP},
every RTS extends to an extensional RTS that has a composite for each pair of composable
transitions.  Such an RTS may be regarded as a category, and in this section
we establish a characterization of the kind of categories that are obtained
from RTS's in this way.

  \input{CategoryWithBoundedPushouts.tex}
  \input{CategoryOfTransitions.tex}

\chapter{RTS Constructions}
\label{rts_constructions_chapter}

This section develops several constructions on residuated transition systems,
including the construction of: an RTS with no transitions (at an arbitrary type),
an RTS with exactly one transition (at any type having at least two elements),
free and fibered (binary) products of RTS's, and an exponential RTS.
These constructions will be used in a subsequent section to construct a cartesian closed
category having residuated transition systems as objects and simulations as arrows.
The natural definitions of the product and exponential constructions on RTS's yield results
at higher types than those of their arguments, but for a cartesian closed category we need
versions of these constructions that produce results at the same type as their arguments.
Since it is not possible in the case of the exponential to carry out such a construction
within HOL (except for finite types), we make use of the ``ZFC in HOL'' axiomatic extension
to HOL to obtain a type having suitable closure properties.  The ZFC in HOL extension
includes definitions of ``smallness'' for sets and types, and we show that each of the
RTS constructions preserves smallness in a suitable sense.  We then show that the small
results (at higher type) of applying the constructions to small arguments can be mapped back,
via functions injective on arrows, to isomorphic copies that ``live'' at the original
argument type.

  \input{RTSConstructions.tex}

\chapter{RTS's in Categories}
\label{rts_in_categories_chapter}

  \input{RTSCategory.tex}
  \input{ConcreteRTSCategory.tex}
  \input{RTSCatx.tex}
  \input{RTSCat.tex}
  \input{RTSCat_Interp.tex}

\chapter{RTS-Enriched Categories}

  \input{RTSEnrichedCategory.tex}

\clearpage
\phantomsection
\addcontentsline{toc}{chapter}{Bibliography}

\bibliographystyle{abbrv}
\bibliography{root}

\end{document}

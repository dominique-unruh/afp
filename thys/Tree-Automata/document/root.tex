\documentclass[11pt,a4paper]{article}
\usepackage[T1]{fontenc}
\usepackage{isabelle,isabellesym}
\usepackage{amssymb}
\usepackage[english]{babel}
\usepackage[only,bigsqcap]{stmaryrd}

% this should be the last package used
\usepackage{pdfsetup}

% urls in roman style, theory text in math-similar italics
\urlstyle{rm}
\isabellestyle{it}

% Tweaks
\newcounter{TTStweak_tag}
\setcounter{TTStweak_tag}{0}
\newcommand{\setTTS}{\setcounter{TTStweak_tag}{1}}
\newcommand{\resetTTS}{\setcounter{TTStweak_tag}{0}}
\newcommand{\insertTTS}{\ifnum\value{TTStweak_tag}=1 \ \ \ \fi}

\renewcommand{\isakeyword}[1]{\resetTTS\emph{\bf\def\isachardot{.}\def\isacharunderscore{\isacharunderscorekeyword}\def\isacharbraceleft{\{}\def\isacharbraceright{\}}#1}}
\renewcommand{\isachardoublequoteopen}{\insertTTS}
\renewcommand{\isachardoublequoteclose}{\setTTS}
\renewcommand{\isanewline}{\mbox{}\par\mbox{}\resetTTS}

\renewcommand{\isamarkupcmt}[1]{\hangindent5ex{\isastylecmt --- #1}}


\begin{document}

\title{Tree Automata}
\author{Peter Lammich}
\maketitle

\begin{abstract}
  This work presents a machine-checked tree automata library for Standard-ML, OCaml and Haskell.
  The algorithms are efficient by using appropriate data structures like RB-trees.
  The available algorithms for non-deterministic automata include membership query, reduction, intersection,
  union, and emptiness check with computation of a witness for non-emptiness.
  
  The executable algorithms are derived from less-concrete, non-executable algorithms using
  data-refinement techniques. The concrete data structures are from the Isabelle Collections Framework.

  Moreover, this work contains a formalization of the class of tree-regular languages and its closure properties under set operations.
\end{abstract}

\clearpage

\tableofcontents

\clearpage

% sane default for proof documents
\parindent 0pt\parskip 0.5ex


\section{Introduction}

This is a formalization of {\em Bounded-Deducibility Security} ({\em BD Security}),
a flexible notion of information-flow security applicable to arbitrary transition systems.
It generalizes Sutherland's classic notion of
nondeducibility~\cite{sutherland-modelOfInf} by factoring in declassification bounds and triggers---whereas nondeducibility
states that, in a system, information cannot flow between specified sources and sinks,
BD security indicates upper bounds for the flow and triggers under which these upper bounds
are no longer guaranteed.

\par \ \par
BD Security was introduced in~\cite{cocon-CAV2014}, where an application to the verification of a conference management called CoCon
system is also presented.
%
The framework is further discussed in detail in \cite{cocon-JAR2021}
and \cite{BDsecurity-ITP2021}.

Other verification case studies of BD Security are discussed
in \cite{cosmed-itp2016,cosmed-jar2018} and \cite{cosmedis-SandP2017}.



% generated text of selected theories
\input{Tree.tex}
\input{Ta.tex}
\input{AbsAlgo.tex}
\input{Ta_impl.tex}

\input{conclusion}

\clearpage

% optional bibliography
\bibliographystyle{abbrv}
\bibliography{root}

\end{document}

%%% Local Variables:
%%% mode: latex
%%% TeX-master: t
%%% End:
